% Options for packages loaded elsewhere
\PassOptionsToPackage{unicode}{hyperref}
\PassOptionsToPackage{hyphens}{url}
%
\documentclass[
]{article}
\usepackage{amsmath,amssymb}
\usepackage{iftex}
\ifPDFTeX
  \usepackage[T1]{fontenc}
  \usepackage[utf8]{inputenc}
  \usepackage{textcomp} % provide euro and other symbols
\else % if luatex or xetex
  \usepackage{unicode-math} % this also loads fontspec
  \defaultfontfeatures{Scale=MatchLowercase}
  \defaultfontfeatures[\rmfamily]{Ligatures=TeX,Scale=1}
\fi
\usepackage{lmodern}
\ifPDFTeX\else
  % xetex/luatex font selection
\fi
% Use upquote if available, for straight quotes in verbatim environments
\IfFileExists{upquote.sty}{\usepackage{upquote}}{}
\IfFileExists{microtype.sty}{% use microtype if available
  \usepackage[]{microtype}
  \UseMicrotypeSet[protrusion]{basicmath} % disable protrusion for tt fonts
}{}
\makeatletter
\@ifundefined{KOMAClassName}{% if non-KOMA class
  \IfFileExists{parskip.sty}{%
    \usepackage{parskip}
  }{% else
    \setlength{\parindent}{0pt}
    \setlength{\parskip}{6pt plus 2pt minus 1pt}}
}{% if KOMA class
  \KOMAoptions{parskip=half}}
\makeatother
\usepackage{xcolor}
\usepackage[margin=1in]{geometry}
\usepackage{graphicx}
\makeatletter
\def\maxwidth{\ifdim\Gin@nat@width>\linewidth\linewidth\else\Gin@nat@width\fi}
\def\maxheight{\ifdim\Gin@nat@height>\textheight\textheight\else\Gin@nat@height\fi}
\makeatother
% Scale images if necessary, so that they will not overflow the page
% margins by default, and it is still possible to overwrite the defaults
% using explicit options in \includegraphics[width, height, ...]{}
\setkeys{Gin}{width=\maxwidth,height=\maxheight,keepaspectratio}
% Set default figure placement to htbp
\makeatletter
\def\fps@figure{htbp}
\makeatother
\setlength{\emergencystretch}{3em} % prevent overfull lines
\providecommand{\tightlist}{%
  \setlength{\itemsep}{0pt}\setlength{\parskip}{0pt}}
\setcounter{secnumdepth}{-\maxdimen} % remove section numbering
\usepackage{float}
\usepackage{sectsty}
\ifLuaTeX
  \usepackage{selnolig}  % disable illegal ligatures
\fi
\IfFileExists{bookmark.sty}{\usepackage{bookmark}}{\usepackage{hyperref}}
\IfFileExists{xurl.sty}{\usepackage{xurl}}{} % add URL line breaks if available
\urlstyle{same}
\hypersetup{
  hidelinks,
  pdfcreator={LaTeX via pandoc}}

\author{}
\date{\vspace{-2.5em}}

\begin{document}

\hypertarget{introduction}{%
\section{Introduction}\label{introduction}}

~~~~~ Organisms and environments are entwined, as the relationship
between organisms and the environments they are embedded within, is in
constant flux through inextricable links and flows of energy. The
relationship between organisms and environments varies through both
space and time and is influenced by a mosaic of dynamic biotic and
abiotic drivers \citep{connell1977mechanisms, menge1987community}.
Organisms are active participants in constructing their environment,
from altering seawater biogeochemistry through physiological processes
to organisms constructing biogenic structures; plenty of studies
demonstrate that organisms influence their environment
\citep{lewontin1983organism,estes1974sea}. Organisms and ecosystems,
while playing an essential role in structuring the physical and
biological environment, are simultaneously governed by the ability to
perform and function under a myriad of complex interactions, as each act
to influence one another non- contemporaneously and contemporaneously
through direct and indirect feedback loops
\citep{levin1974disturbance,paine1980food}. Organisms must
physiologically cope with the conditions of the environment they are
situated within, which inevitably influences community structure and
populations \citep{bozinovic2015physiological}. Further, the biogenic
structures created by organisms are highly dependent upon the
environmental regime the organism develops in, living beyond the life of
the organism itself; organisms are thus simultaneously creators and
products of the environment. In a geological epoch of rapid ecological
change, it is increasingly imperative to understand how and the extent
to which organisms can respond and perform to abiotic drivers and how
the legacy of the structures (e.g., shells, reefs) that organisms create
may influence other species indirectly. Understanding physiological
responses, interactions, and constraints of marine organisms to
anthropogenic climate change is perhaps the sine qua non for
understanding the changes between marine organisms and the ecosystems
they construct. This research intends to inform how the changing oceanic
environment may affect organismal physiology by teasing apart the
relationship between environmental drivers and physiological
performance. This research also intends to elucidate how changes within
physiological processes in one organism may have indirect effects on
other species long after the organism persists. In this regard, the fate
of organisms is intertwined as the abundance and growth of one species
codetermines the other, illustrating that organisms are directly or
indirectly the subjects and objects of ecological change
\citep{lewontin1983organism}.

\hypertarget{organisms-as-the-objects-of-ecological-change}{%
\section{ORGANISMS AS THE OBJECTS OF ECOLOGICAL
CHANGE}\label{organisms-as-the-objects-of-ecological-change}}

\hypertarget{organisms-as-the-subjects-of-ecological-change}{%
\section{Organisms as the Subjects of Ecological
Change}\label{organisms-as-the-subjects-of-ecological-change}}

~~~~~Environments are governed by natural spatiotemporal variation of
abiotic and biotic drivers that, in turn, influence the structure and
processes of communities, drive ecological change, and create a mosaic
of microhabitats
\citep{connell1961influence, stenseth2002ecological, kroeker2017embracing}.
For example, within marine ecosystems, the combination of oceanographic
processes and local coastal geography may create an array of patterns
and variability in abiotic drivers such as temperature, flow, pH,
dissolved oxygen, etc., that may impact the structure and processes of
ecological communities on distances ranging from microscale to
macroscale \citep{deser2010sea, hofmann2010living}. Such heterogeneous
patterns are naturally occurring and create complex gradients that shape
ecological communities and influence physiological processes within
individual organisms \citep{helmuth2006mosaic}. However, due to the
connotation of stress as a negative response and the ability for
organisms to adapt and evolve to changing conditions over time, the term
driver has been utilized to describe an environmental parameter that
influences organisms and environments across a spectrum ranging from
enhancing, optimal, or stressful conditions
\citep{cote2016interactions, boyd2012understanding}. Many organisms have
evolved to withstand complex and variable environmental gradients
through physiological mechanisms such as phenotypic plasticity and
acclimatization \citep{hofmann2010living, tomanek2002heat}. According to
the metabolic theory of ecology, environmental gradients and changes in
abiotic factors may result in physiological trade-offs due to the
alterations within the energetic partitioning of an organism's
metabolism \citep{portner2008physiology, brown2004metabolic}. The
physiological processes of metabolism are the total sum of biological
and chemical processes in converting energetic resources and materials
into biomass and activity \citep{brown2004metabolic}. Comparing
physiological responses to gradients of abiotic drivers may allow us to
quantify and compare the tolerance limits of organisms
\citep{somero2002thermal, silbiger2019comparative}. The role of biotic
and abiotic drivers in influencing metabolic processes has been of
primary interest to the field of ecology as changes in metabolism
directly affect the survival, behavior, and energy requirements of
organisms, thereby impacting fitness and ecosystem function
\citep{carey2016sea}.

~~~~~ Temperature and pH are important for determining physiological
processes and metabolic rates for marine organisms and thereby play a
large role in affecting the functioning and physiology of ecosystems
\citep{woodwell1970effects}. Temperature is the key driver in
determining physiological rates of organisms as the kinetic energy of
biochemical reactions is temperature dependent
\citep{levins1968evolution, somero2002thermal, portner2012integrating}.
Biological processes such as organismal and ecological interactions are
also strongly influenced by temperature
\citep{hochachka2002biochemical}. The relationship between temperature
and body-size exemplifies this as organisms develop faster yet decrease
in size under elevated temperatures \citep{elahi2020historical}.
Metabolic rates are strongly influenced by an organism's body size and
temperature and are subject to change due to changes in abiotic drivers
and the natural variability of drivers
\citep{brown2004metabolic, oconnor2007temperature}. Further, organisms
adapt to local temperatures to match optimal conditions for
physiological processes and acclimatize to a range around these values
\citep{sinclair2016can}. Any range too far beyond the ability of an
organism to acclimatize influences survival, fitness, and population
densities \citep{hochachka2002biochemical}. Studies have shown the
influence of sea surface temperature on metabolic processes such as
growth, feeding, reproduction, and influencing the range of species
distributions
\citep{kordas2011community, sanford2002feeding, pinsky2013marine}.
However, it is essential to note that temperature is not the only driver
of biological processes and temperature has interactive effects with
other abiotic drivers \citep{darling2008quantifying}. pH is also an
important abiotic driver that impacts the physiological performance of
marine organisms and influences the biogenic structures that organisms
create \citep{hofmann2010living}. pH plays a vital role in metabolic
processes due to its effect on biochemical pathways and internal
acid-base balance \citep{gaylord2015ocean}. For example, low pH is often
associated with elevated metabolic rates due to the increase in
energetic costs in creating calcified structures such as the formation
of shells in mollusks or the skeletons of corals and echinoderms
\citep{doney2009ocean, spalding2017energetic}. Due to differences in the
energetic costs associated with calcification, there are significant
differences in the ability to control acid-base regulation between
species \citep{doney2009ocean}. Consequently, changes in physicochemical
parameters of the environment affect species differently, impact the
interaction between species and, in turn, affect the structures of
ecological communities; therefore, studying how differences between
abiotic drivers affect organismal physiology will have ecosystem-level
implications \citep{tomanek2002physiological, barclay2019variation}.

\hypertarget{organismal-physiology-in-a-changing-environment}{%
\section{ORGANISMAL PHYSIOLOGY IN A CHANGING
ENVIRONMENT}\label{organismal-physiology-in-a-changing-environment}}

~~~~~ As the atmospheric carbon dioxide (CO2) concentration continues to
surpass the limits of the earth system, marine organisms will be forced
to endure profound transformations of the environment, from shifts in
temperature to altered geochemistry (richardson2023earth,
portner2008physiology\}. Ocean warming (OW) and ocean acidification (OA)
represent two of the most significant changes occurring in marine
ecosystems across the globe, both driven by the unremitted rise of
anthropogenic-induced carbon dioxide emissions. OW and OA are not
isolated phenomena; they share a common origin, and in a rapidly
changing world, their combined impacts on organismal physiology
necessitate special attention as multiple drivers of change may act
interactively \citep{cote2016interactions}. Since the beginning of the
20th century, the global mean sea surface temperature (SST) has
increased by 0.88 {[}0.68--1.01{]} degree C, and is further projected to
warm by 2.89 degree C {[}2.01--4.07 degree C{]} at the end of the
century, which surpasses the thermal tolerance limits of many marine
species (following the representative concentration pathway 8.5 emission
scenario)
\citep{kikstra2022ipcc, fox2021ocean, bay2017genomic, somero2010physiology}.
Concurrently, the ocean has absorbed \textasciitilde30\% of
anthropogenic CO2 \citep{feely2004impact}, altering the carbonate
chemistry of seawater through a decrease in the concentration of
carbonate ions \(\mathrm{CO_3^{2-}}\) and a decline in seawater pH
\citep{feely2004impact}. Mean surface ocean pH values have declined by
0.1 units since the pre-industrial era, with a further projected
diminution of 0.1 - 0.4 units by the end of the century
\citep{change2014impacts, orr2005anthropogenic}, posing a unique threat
to calcifying marine organisms. Consequently, the impacts of OW and OA
will not be consistent across geographic regions, leading to
differential effects that will modify already variable spatial and
temporal environments. Building a mechanistic understanding of how the
combined impacts of ocean warming and acidification affect marine
organisms is integral for reliable projections of how climate change may
continue to affect marine organisms.

~~~~~ Coastal marine organisms frequently encounter a wide range of
temperatures and experience fluctuations in biogeochemistry, resulting
from temporal variations, such as tidal and seasonal cycles. The rocky
intertidal system is one such system that is known for its variable
conditions on both temporal and spatial scales, making them a model
ecosystem for understanding how organisms interact and respond to change
\citep{connell1961influence, paine1969pisaster, kwiatkowski2016nighttime, jellison2022low}.
Organisms within the rocky intertidal zone must contend with alternating
periods of immersion and emersion of tidal fluctuations, which commonly
lead to large variations in temperature, oxygen availability, and pH,
\citep{denny2001physical, helmuth2002climate}. Of these naturally
occurring changes, thermal variability within the intertidal zone is
believed to be a dominant driver in structuring the vertical and
latitudinal distribution patterns by limiting upper zonation through
abiotic stress and lower zonation through biotic influence
\citep{helmuth2006mosaic, somero2002thermal, somero2010physiology, connell1961influence}.
Daily temperature fluctuations are drastic enough to elevate the body
temperatures of marine organisms by more than 20 degree C during a tidal
emersion event \citep{Helmuth1999}. Furthermore, changes in pH within
tidepools may exceed 1 unit when nighttime respiration rates exceed
photosynthetic rates
\citep{jellison2016ocean, kwiatkowski2016nighttime}. Such highly
variable abiotic changes are naturally occurring and create complex
gradients that shape ecological communities and influence physiological
processes within individual organisms \citep{helmuth2006mosaic}. Given
that organisms within the intertidal zone simultaneously face drastic
fluctuations from abiotic drivers, and experience conditions far beyond
what is expected in the future, understanding organismal performance in
these ecosystems may provide a window for looking toward the future.

~~~~~ The role of biotic and abiotic drivers in influencing metabolic
processes has been of primary interest to the field of ecology as
changes in metabolism directly affect the survival, behavior, and energy
requirements of organisms, thereby impacting organism and ecosystem
function \citep{carey2016sea}. Physiological processes are heavily
influenced by environmental factors, and many marine organisms undergo
biological responses to natural diel variability present within
environments \citep{hofmann2010living}. Temperature is the primary
environmental driver regulating physiological rates of ectothermic
organisms, as kinetic energy of biochemical reactions are temperature
dependent
\citep{levins1968evolution, huey1979integrating, hochachka2002biochemical}.
Further, temperature is the key determinant in the regulating rates of
biological processes, ranging from metabolic rates
\citep{gillooly2001effects} to species interactions
\citep{sanford1999regulation}, such as growth, feeding, reproduction,
and determining the range of species distributions
\citep{kordas2011community, sanford2002feeding, pinsky2013marine}. The
physiological processes of metabolism are the total sum of biological
and chemical processes in converting energetic resources and materials
into biomass and activity \citep{brown2004metabolic}. Changes in pH also
play a vital role in metabolic processes due to its effect on
biochemical pathways and internal acid-base balance
\citep{gaylord2015ocean}. Specifically, declines in seawater carbonate
ions and pH attributed to OA are strongly correlated to decreases in
calcification and growth rates of many marine organisms
\citep{kroeker2013impacts}. Due to species-specific differences in the
energetic costs associated with calcification, there are significant
differences in the ability to control internal acid-base regulation
between species \citep{doney2009ocean}. Ultimately, changes in the
environment that lead to alterations in organismal energetic
requirements will scale up to affect the processes of ecological
communities; therefore, studying how multiple abiotic factors affect
organismal physiology has ecosystem-level implications
\citep{tomanek2002heat, barclay2019variation, kroeker2022ecological}.

\end{document}
