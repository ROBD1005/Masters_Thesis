% Options for packages loaded elsewhere
\PassOptionsToPackage{unicode}{hyperref}
\PassOptionsToPackage{hyphens}{url}
%
\documentclass[
  11pt,
]{article}
\usepackage{amsmath,amssymb}
\usepackage{lmodern}
\usepackage{iftex}
\ifPDFTeX
  \usepackage[T1]{fontenc}
  \usepackage[utf8]{inputenc}
  \usepackage{textcomp} % provide euro and other symbols
\else % if luatex or xetex
  \usepackage{unicode-math}
  \defaultfontfeatures{Scale=MatchLowercase}
  \defaultfontfeatures[\rmfamily]{Ligatures=TeX,Scale=1}
\fi
% Use upquote if available, for straight quotes in verbatim environments
\IfFileExists{upquote.sty}{\usepackage{upquote}}{}
\IfFileExists{microtype.sty}{% use microtype if available
  \usepackage[]{microtype}
  \UseMicrotypeSet[protrusion]{basicmath} % disable protrusion for tt fonts
}{}
\makeatletter
\@ifundefined{KOMAClassName}{% if non-KOMA class
  \IfFileExists{parskip.sty}{%
    \usepackage{parskip}
  }{% else
    \setlength{\parindent}{0pt}
    \setlength{\parskip}{6pt plus 2pt minus 1pt}}
}{% if KOMA class
  \KOMAoptions{parskip=half}}
\makeatother
\usepackage{xcolor}
\usepackage[left = 2.5cm, right = 2cm, top = 2cm, bottom =
2cm]{geometry}
\usepackage{color}
\usepackage{fancyvrb}
\newcommand{\VerbBar}{|}
\newcommand{\VERB}{\Verb[commandchars=\\\{\}]}
\DefineVerbatimEnvironment{Highlighting}{Verbatim}{commandchars=\\\{\}}
% Add ',fontsize=\small' for more characters per line
\usepackage{framed}
\definecolor{shadecolor}{RGB}{248,248,248}
\newenvironment{Shaded}{\begin{snugshade}}{\end{snugshade}}
\newcommand{\AlertTok}[1]{\textcolor[rgb]{0.94,0.16,0.16}{#1}}
\newcommand{\AnnotationTok}[1]{\textcolor[rgb]{0.56,0.35,0.01}{\textbf{\textit{#1}}}}
\newcommand{\AttributeTok}[1]{\textcolor[rgb]{0.77,0.63,0.00}{#1}}
\newcommand{\BaseNTok}[1]{\textcolor[rgb]{0.00,0.00,0.81}{#1}}
\newcommand{\BuiltInTok}[1]{#1}
\newcommand{\CharTok}[1]{\textcolor[rgb]{0.31,0.60,0.02}{#1}}
\newcommand{\CommentTok}[1]{\textcolor[rgb]{0.56,0.35,0.01}{\textit{#1}}}
\newcommand{\CommentVarTok}[1]{\textcolor[rgb]{0.56,0.35,0.01}{\textbf{\textit{#1}}}}
\newcommand{\ConstantTok}[1]{\textcolor[rgb]{0.00,0.00,0.00}{#1}}
\newcommand{\ControlFlowTok}[1]{\textcolor[rgb]{0.13,0.29,0.53}{\textbf{#1}}}
\newcommand{\DataTypeTok}[1]{\textcolor[rgb]{0.13,0.29,0.53}{#1}}
\newcommand{\DecValTok}[1]{\textcolor[rgb]{0.00,0.00,0.81}{#1}}
\newcommand{\DocumentationTok}[1]{\textcolor[rgb]{0.56,0.35,0.01}{\textbf{\textit{#1}}}}
\newcommand{\ErrorTok}[1]{\textcolor[rgb]{0.64,0.00,0.00}{\textbf{#1}}}
\newcommand{\ExtensionTok}[1]{#1}
\newcommand{\FloatTok}[1]{\textcolor[rgb]{0.00,0.00,0.81}{#1}}
\newcommand{\FunctionTok}[1]{\textcolor[rgb]{0.00,0.00,0.00}{#1}}
\newcommand{\ImportTok}[1]{#1}
\newcommand{\InformationTok}[1]{\textcolor[rgb]{0.56,0.35,0.01}{\textbf{\textit{#1}}}}
\newcommand{\KeywordTok}[1]{\textcolor[rgb]{0.13,0.29,0.53}{\textbf{#1}}}
\newcommand{\NormalTok}[1]{#1}
\newcommand{\OperatorTok}[1]{\textcolor[rgb]{0.81,0.36,0.00}{\textbf{#1}}}
\newcommand{\OtherTok}[1]{\textcolor[rgb]{0.56,0.35,0.01}{#1}}
\newcommand{\PreprocessorTok}[1]{\textcolor[rgb]{0.56,0.35,0.01}{\textit{#1}}}
\newcommand{\RegionMarkerTok}[1]{#1}
\newcommand{\SpecialCharTok}[1]{\textcolor[rgb]{0.00,0.00,0.00}{#1}}
\newcommand{\SpecialStringTok}[1]{\textcolor[rgb]{0.31,0.60,0.02}{#1}}
\newcommand{\StringTok}[1]{\textcolor[rgb]{0.31,0.60,0.02}{#1}}
\newcommand{\VariableTok}[1]{\textcolor[rgb]{0.00,0.00,0.00}{#1}}
\newcommand{\VerbatimStringTok}[1]{\textcolor[rgb]{0.31,0.60,0.02}{#1}}
\newcommand{\WarningTok}[1]{\textcolor[rgb]{0.56,0.35,0.01}{\textbf{\textit{#1}}}}
\usepackage{graphicx}
\makeatletter
\def\maxwidth{\ifdim\Gin@nat@width>\linewidth\linewidth\else\Gin@nat@width\fi}
\def\maxheight{\ifdim\Gin@nat@height>\textheight\textheight\else\Gin@nat@height\fi}
\makeatother
% Scale images if necessary, so that they will not overflow the page
% margins by default, and it is still possible to overwrite the defaults
% using explicit options in \includegraphics[width, height, ...]{}
\setkeys{Gin}{width=\maxwidth,height=\maxheight,keepaspectratio}
% Set default figure placement to htbp
\makeatletter
\def\fps@figure{htbp}
\makeatother
\setlength{\emergencystretch}{3em} % prevent overfull lines
\providecommand{\tightlist}{%
  \setlength{\itemsep}{0pt}\setlength{\parskip}{0pt}}
\setcounter{secnumdepth}{5}
\usepackage{float}
\usepackage{sectsty}
\usepackage{paralist}
\usepackage{setspace}\spacing{1.5}
\usepackage{fancyhdr}
\usepackage{lastpage}
\usepackage{dcolumn}
\usepackage{natbib}\bibliographystyle{agsm}
\usepackage[nottoc, numbib]{tocbibind}
\ifLuaTeX
  \usepackage{selnolig}  % disable illegal ligatures
\fi
\IfFileExists{bookmark.sty}{\usepackage{bookmark}}{\usepackage{hyperref}}
\IfFileExists{xurl.sty}{\usepackage{xurl}}{} % add URL line breaks if available
\urlstyle{same} % disable monospaced font for URLs
\hypersetup{
  pdftitle={Masters Thesis},
  pdfauthor={Robert J. Dellinger},
  hidelinks,
  pdfcreator={LaTeX via pandoc}}

\title{Masters Thesis}
\author{\href{https://robdellinger.com}{Robert J. Dellinger}}
\date{2023-01-01}

\begin{document}
\maketitle

\newpage

\allsectionsfont{\centering}
\subsectionfont{\raggedright}
\subsubsectionfont{\raggedright}

\begin{centering}

\vspace{3cm}

\vspace{1cm}


\includegraphics[width=0.3\linewidth]{Images/university_logo} 

\vspace{0.5cm}
\Large
{\bf CALIFORNIA STATE UNIVERSITY, NORTHRIDGE}

\Large
{\bf Department of Biology}
\vspace{1cm}
\Large

\doublespacing
{\bf ORGANISMS AS THE SUBJECTS AND OBJECTS OF ECOLOGICAL CHANGE \\
A Study Regarding The Indirect Effects of Ocean Acidification and Warming
on Two Intertidal Species\\}


\normalsize
\singlespacing
By

\vspace{0.5 cm}

\Large
{\bf Robert J. Dellinger}

\vspace{0.5 cm}
{Nyssa J. Silbiger, Ph.D.}
{Peter J. Edmunds, Ph.D.}
{Kerry J. Nickols, Ph.D.}


\vspace{1.5 cm}
Thesis submitted in partial fulfillment \\
of the requirements for the degree of  \\
Master of Science in Biology \\
\vspace{1.5 cm}

\normalsize
mm/yyyy

   
\newpage








\section*{Signature Page}

 The signature page involves the electronic version that does not require actual signatures. The electronic version of the document must include an unsigned formatted signature page.
\pagenumbering{roman}
\newpage







\section*{Preface}

 
\newpage







\section*{Acknowledgements}

You may want to include an acknowledgment of help received from particular people or a brief note of dedication. Such notes should follow at this point. If you are using published material in the main body of the manuscript, you must enclose an acknowledgment indicating where your work has been published. Do not include acknowledgments at the end of each chapter. This page should be numbered using lower case Roman numerals. If you choose to include both Acknowledgments and Dedication, they should be on separate pages, with the Acknowledgment page appearing first. The text of these pages must be double-spaced. 

\centering
\raggedright
\newpage
\tableofcontents

Each thesis or dissertation is expected to have a table of contents for the convenience of the reader. There is no specific format that students must follow in this case. The table of contents should be easy to read, consistent and have page numbers listed on the right side of the page. The table of contents must include page numbers for the preliminary pages, with the exception of the table of contents. A typical table of contents does not exceed 3 pages.
\newpage


\begin{centering}
{\bf Abstract}
\end{centering}
\spacing{1.5}

An abstract does not typically exceed 350 words -- approximately one and
one-half pages doubled-spaced. The abstract should state the research
problem briefly, describe the methods and procedures used in gathering
data or studying the problem, and give a condensed summary of the
findings of the study. The abstract heading, prepared according to
sample page in the Appendix E, must have ``Abstract'' as the main title,
include the title of the thesis/dissertation, your name as it appears on
your University records, the degree, and your graduate program. \newpage

\hypertarget{introduction}{%
\section{Introduction}\label{introduction}}

\begin{itemize}
\tightlist
\item
  introduce the reader to the subject area and clarify the knowledge gap
  that the dissertation research will fill.
\item
  set the context for the dissertation by reviewing the relevant
  literature.
\item
  include relevant references to general (theoretical papers and
  reviews) and specific (specific to the particular question addressed)
  literature, to justify the research that has been undertaken and
  define the questions being addressed.
\item
  state the primary research questions and hypotheses in the final
  paragraph.
\item
  follow an `inverted triangle' format, progressing from general
  scientific ideas and why they matter to the specific research
  questions addressed in the dissertation project.
\end{itemize}

\emph{The introduction should not be just a `Literature Review'.}

\begin{centering}
INTRODUCTION
\end{centering}

Organisms and environments are entwined, as the relationship between
organisms and the environments they are embedded within, is in constant
flux through inextricable links and flows of energy. The relationship
between organisms and environments varies through both space and time
and is influenced by a mosaic of dynamic biotic and abiotic drivers
(Connell, 1977; Menge \& Sutherland, 1987). Organisms are active
participants in constructing their environment, from altering seawater
biogeochemistry through physiological processes to organisms
constructing biogenic structures; plenty of studies demonstrate that
organisms influence their environment (Lewontin, 1983; Estes \&
Palmisano, 1974; Silbiger et al., 2018). Organisms and ecosystems, while
playing an essential role in structuring the physical and biological
environment, are simultaneously governed by the ability to perform and
function under a myriad of complex interactions, as each act to
influence one another non- contemporaneously and contemporaneously
through direct and indirect feedback loops (Levin \& Paine, 1974; Paine,
1981). Organisms must physiologically cope with the conditions of the
environment they are situated within, which inevitably influences
community structure and populations (Bozinovic \& Pörtner, 2015).
Further, the biogenic structures created by organisms are highly
dependent upon the environmental regime the organism develops in, living
beyond the life of the organism itself; organisms are thus
simultaneously creators and products of the environment. In a geological
epoch of rapid ecological change, it is increasingly imperative to
understand how and the extent to which organisms can respond and perform
to abiotic drivers and how the legacy of the structures (e.g., shells,
reefs) that organisms create may influence other species indirectly.
Understanding physiological responses, interactions, and constraints of
marine organisms to anthropogenic climate change is perhaps the sine qua
non for understanding the changes between marine organisms and the
ecosystems they construct. This research intends to inform how the
changing oceanic environment may affect organismal physiology by teasing
apart the relationship between environmental drivers and physiological
performance. This research also intends to elucidate how changes within
physiological processes in one organism may have indirect effects on
other species long after the organism persists. In this regard, the fate
of organisms is intertwined as the abundance and growth of one species
codetermines the other, illustrating that organisms are directly or
indirectly the subjects and objects of ecological change (Lewontin,
1983).

\begin{centering}
ORGANISMS AS THE OBJECTS OF ECOLOGICAL CHANGE
\end{centering}

Environments are governed by natural spatiotemporal variation of abiotic
and biotic drivers that, in turn, influence the structure and processes
of communities, drive ecological change, and create a mosaic of
microhabitats (Connell, 1961; Stenseth et al., 2002; Kroeker et al.,
2017). For example, within marine ecosystems, the combination of
oceanographic processes and local coastal geography may create an array
of patterns and variability in abiotic drivers such as temperature,
flow, pH, dissolved oxygen, etc., that may impact the structure and
processes of ecological communities on distances ranging from microscale
to macroscale (Deser, et al., 2010; Hoffman et al., 2010). Such
heterogeneous patterns are naturally occurring and create complex
gradients that shape ecological communities and influence physiological
processes within individual organisms (Helmuth et al., 2006). However,
due to the connotation of stress as a negative response and the ability
for organisms to adapt and evolve to changing conditions over time, the
term driver has been utilized to describe an environmental parameter
that influences organisms and environments across a spectrum ranging
from enhancing, optimal, or stressful conditions (Côté et al., 2016;
Boyd and Hutchins, 2012). Many organisms have evolved to withstand
complex and variable environmental gradients through physiological
mechanisms such as phenotypic plasticity and acclimatization (Hoffman \&
Togham, 2010; Tomanek, 2002). According to the metabolic theory of
ecology, environmental gradients and changes in abiotic factors may
result in physiological trade-offs due to the alterations within the
energetic partitioning of an organism's metabolism (Pörtner, 2008; Brown
et al., 2004). The physiological processes of metabolism are the total
sum of biological and chemical processes in converting energetic
resources and materials into biomass and activity (Brown et al., 2004).
Comparing physiological responses to gradients of abiotic drivers may
allow us to quantify and compare the tolerance limits of organisms
(Somero, 2002; Silbiger et al., 2019). The role of biotic and abiotic
drivers in influencing metabolic processes has been of primary interest
to the field of ecology as changes in metabolism directly affect the
survival, behavior, and energy requirements of organisms, thereby
impacting fitness and ecosystem function (Carey, Harianto, \& Byrne,
2016).

Temperature and pH are important for determining physiological processes
and metabolic rates for marine organisms and thereby play a large role
in affecting the functioning and physiology of ecosystems (Woodwell,
1970). Temperature is the key driver in determining physiological rates
of organisms as the kinetic energy of biochemical reactions is
temperature dependent (Levins, 1968; Somero 2002, Pörtner, 2012).
Biological processes such as organismal and ecological interactions are
also strongly influenced by temperature (Hochachka and Somero, 2002).
The relationship between temperature and body-size exemplifies this as
organisms develop faster yet decrease in size under elevated
temperatures (Elahi, 2020). Metabolic rates are strongly influenced by
an organism's body size and temperature and are subject to change due
tochanges in abiotic drivers and the natural variability of drivers
(Brown et al., 2004; O'Connor, 2007). Further, organisms adapt to local
temperatures to match optimal conditions for physiological processes and
acclimatize to a range around these values (Sinclair et al., 2016). Any
range too far beyond the ability of an organism to acclimatize
influences survival, fitness, and population densities (Hochachka and
Somero, 2002). Studies have shown the influence of sea surface
temperature on metabolic processes such as growth, feeding,
reproduction, and influencing the range of species distributions (Kordas
et al., 2011; Sanford, 2002; Pinsky et al., 2013). However, it is
essential to note that temperature is not the only driver of biological
processes and temperature has interactive effects with other abiotic
drivers (Darling and Côté 2008). pH is also an important abiotic driver
that impacts the physiological performance of marine organisms and
influences the biogenic structures that organisms create (Hoffman and
Togham, 2010). pH plays a vital role in metabolic processes due to its
effect on biochemical pathways and internal acid-base balance (Gaylord
et al., 2015). For example, low pH is often associated with elevated
metabolic rates due to the increase in energetic costs in creating
calcified structures such as the formation of shells in mollusks or the
skeletons of corals and echinoderms (Doney et al., 2009; Spalding et
al., 2017). Due to differences in the energetic costs associated with
calcification, there are significant differences in the ability to
control acid-base regulation between species (Doney et al., 2009).
Consequently, changes in physicochemical parameters of the environment
affect species differently, impact the interaction between species and,
in turn, affect the structures of ecological communities; therefore,
studying how differences between abiotic drivers affect organismal
physiology will have ecosystem-level implications (Tomanek \& Helmuth,
2002; Gaylord 2019).

\begin{centering}
ORGANISMAL PHYSIOLOGY IN A CHANGING ENVIRONMENT
\end{centering}

Anthropogenic induced changes to the carbon cycle, as a result of
increasing carbon dioxide (CO2) emissions, are influencing seawater
temperature and the acidity of the ocean (decrease in pH) from the
dissolution of CO2 altering seawater carbonate chemistry. Since the
beginning of the 20th century, marine ecosystems have experienced an
average sea surface temperature increase of 1.5° C which is expected to
rise by 1--4 °C by the end of the 21st century (IPCC, 2022). The highly
variable mosaic of temperature regimes will influence the ability for
organisms to respond and adapt to future temperature changes due to the
influence of environmental history on organismal responses to stress
(Safaie, et al.~2018). However, these projected scenarios will surpass
the thermal tolerance limit of many marine species (Bay et al., 2017;
Somnero, 2010). Further, the ocean has absorbed nearly 30\% of
anthropogenic CO2 (Sabine et al., 2004), declining the global average pH
by 0.1 units, and is expected to decrease average pH by another 0.3-0.4
units by the end of the century (Caldeira \& Wickett, 2003). It is
imperative to note that future changes in pH, much like temperature,
will be highly variable and differ based on local oceanographic
phenomena and conditions, for example, the intensification of upwelling
in certain regions (Garcia-Reyes \& Largier, 2012; Bakun, 1990). The
expected future changes in seawater carbonate physicochemical parameters
will impact marine organisms by altering the range and variability of pH
and temperature that organisms experience (Bakun, 1990; IPCC, 2022).
Further, studies have shown that changes in pH will have profound
effects on the ability of calcifying organisms to produce their
skeletons and shells due to an increase in the energy required to
maintain calcified structures in acidified seawater (Kroeker et al.,
2010; Gaylord et al., 2015). Calcified structures may be constructed
weaker, or calcification may be maintained at the expense of other
metabolic processes (Gaylord, et al.~2015). These structures, which are
products of the environmental history that they were produced in,
outlive the organisms that create them and influence other species, such
as the gastropod constructed shells that hermit crabs inhabit or the
structural complexity that corals create through the accretion of
calcium carbonate in skeleton construction (Laidre, 2011; Graham and
Nash, 2013). Consequently, ocean acidification and warming, as drivers
of ecological change attributed to climate change, co-occur and should
be studied together due to potentially interactive effects within and
between species.

Organisms are adapted to cope with a natural range of abiotic
conditions, yet anthropogenic climate change may outpace organismal
physiological capacities and may also act interactively to result in
``ecological surprises'\,' (sense, Paine et al., 1969). For example, the
cumulative impact of two drivers acting together may interact to be
equivalent to their sum, known as an additive effect, less than their
additive effect, which is known as an antagonistic interaction, or
greater than their additive effect, known as a synergistic interaction
(Côté et al., 2016). For decades anti-racist and feminist scholars have
provided critical insight into the ways in we must think about isolated
phenomena as intersectional due to their complex interactions (Davis,
1983; Crenshaw, 1989). The field of marine biology requires this
paradigm shift that embraces the interactions between stressors to grasp
the complexity of the future. Combined, these drastic changes in abiotic
drivers will continue to act in conjecture with one another and could
potentially ameliorate or exacerbate impacts on organismal physiological
processes and reverberate the effects of ecological change through
ecosystems (Kroeker, Kordas, \& Harley, 2017).

Despite the increasing emphasis on multi-driver and multi-species
studies, a mechanistic understanding of the nonlinear responses of
multiple confounding stressors of future scenarios remains a knowledge
gap. (Kroeker, Kordas, \& Harley, 2017). The elevated sense of urgency
involved with these global threats contributes to the need for a more
nuanced understanding of the impact of multiple stressor interactions on
organismal and ecosystem processes (Côté et al., 2016). The response of
organisms to climate change, survival, and fitness depends on
physiological trade-offs, which are essentially changed within the
allocation of an organism's energy for different biological functions
and demands for resources. Considering that metabolic processes respond
differently to multiple environmental drivers and that physiological
systems within and between organisms differ, it is imperative to tease
apart the metabolic rates. Organisms may respond to changing abiotic
drivers by altering their energetic allocation or energetic intake, such
as altering consumption rates or growth (O'Connor 2009). For example,
the metabolic theory of ecology predicts an increase in consumption
rates with increasing temperature (O'Connor 2009). These alterations
within an individual species' physiological performance are significant
because they can scale up to affect ecosystem function (Post et al.,
1999). Building a mechanistic understanding regarding how the combined
impacts of ocean warming and acidification affect marine organisms is
integral for reliable projections of how climate change may continue to
affect marine organisms.

\pagenumbering{arabic}
\newpage

\hypertarget{methods}{%
\section{Methods}\label{methods}}

\vspace{0.5cm}

Write your methods here. In this tutorial you can use this already made
file\textbackslash to add examples of figures and tables and explore
knitr and kableExtra functionalities! \newpage

\hypertarget{results}{%
\section{Results}\label{results}}

Some more guidlines from the School of Geosciences.

This section should summarise the findings of the research referring to
all figures, tables and statistical results (some of which may be placed
in appendices). - include the primary results, ordered logically - it is
often useful to follow the same order as presented in the methods. -
alternatively, you may find that ordering the results from the most
important to the least important works better for your project. - data
should only be presented in the main text once, either in tables or
figures; if presented in figures, data can be tabulated in appendices
and referred to at the appropriate point in the main text.

\textbf{Often, it is recommended that you write the results section
first, so that you can write the methods that are appropriate to
describe the results presented. Then you can write the discussion next,
then the introduction which includes the relevant literature for the
scientific story that you are telling and finally the conclusions and
abstract -- this approach is called writing backwards.} \newpage

\hypertarget{discussion}{%
\section{Discussion}\label{discussion}}

the purpose of the discussion is to summarise your major findings and
place them in the context of the current state of knowledge in the
literature. When you discuss your own work and that of others, back up
your statements with evidence and citations. - The first part of the
discussion should contain a summary of your major findings (usually 2 --
4 points) and a brief summary of the implications of your findings.
Ideally, it should make reference to whether you found support for your
hypotheses or answered your questions that were placed at the end of the
introduction. - The following paragraphs will then usually describe each
of these findings in greater detail, making reference to previous
studies. - Often the discussion will include one or a few paragraphs
describing the limitations of your study and the potential for future
research. - Subheadings within the discussion can be useful for
orienting the reader to the major themes that are addressed. \newpage

\hypertarget{conclusion}{%
\section{Conclusion}\label{conclusion}}

The conclusion section should specify the key findings of your study,
explain their wider significance in the context of the research field
and explain how you have filled the knowledge gap that you have
identified in the introduction. This is your chance to present to your
reader the major take-home messages of your dissertation research. It
should be similar in content to the last sentence of your summary
abstract. It should not be a repetition of the first paragraph of the
discussion. They can be distinguished in their connection to broader
issues. The first paragraph of the discussion will tend to focus on the
direct scientific implications of your work (i.e.~basic science,
fundamental knowledge) while the conclusion will tend to focus more on
the implications of the results for society, conservation, etc. \newpage

\#citep\{breton\_encounter\_2006\} \#cite in text \#

\bibliography{bibliography}

\#command to add the references page\ldots{}

A Bibliography, Works Cited or Reference Section should follow the text
and notes and will always begin on a new page. References are
single-spaced with a blank line between each entry. The order of the
bibliography and its format is a matter for discussion with your
committee. However, for the ease of the reader it is preferable to have
one complete alphabetical listing at the end of the manuscript. When
citing electronic sources, a digital object identifier (DOI), a unique
alphanumeric string assigned to identify content and provide a
persistent link to its location on the website is required. A retrieval
date (month and year) is only needed in the reference list for
nonjournal instances where material might change at a later date.
\newpage

\hypertarget{appendixces}{%
\section{Appendix(ces)}\label{appendixces}}

A last section may contain supporting data for the text in the form of
one or more appendices. Appendices should be placed after the
bibliography. The appendices must fall within the margin requirements
and may be single-spaced if necessary. The ETD website gives students
the option to upload ``Supporting Files'' in addition to the
thesis/dissertation. Supplemental files can include large appendix type
material, videos, images, audio files, PowerPoint presentations, and any
other file type, which will not be embedded into the main thesis
document.

\hypertarget{appendix-a-additional-tables}{%
\subsection{Appendix A: additional
tables}\label{appendix-a-additional-tables}}

Insert content for additional tables here.

\newpage

\hypertarget{appendix-b-additional-figures}{%
\subsection{Appendix B: additional
figures}\label{appendix-b-additional-figures}}

Insert content for additional figures here.

\newpage

\hypertarget{appendix-c-code}{%
\subsection{Appendix C: code}\label{appendix-c-code}}

Insert code (if any) used during your dissertation work here.

\begin{Shaded}
\begin{Highlighting}[]
\CommentTok{\#Converting a .csv bibliography (google scholar) to a .bib bibliography }

\CommentTok{\#path \textless{}{-} here("Bibliography", "bibliography.csv")}

\CommentTok{\#bib \textless{}{-} revtools::read\_bibliography(path, return\_df = FALSE)}

\CommentTok{\#revtools::write\_bibliography(bib,}
\CommentTok{\#filename = "bibliography.bib",}
\CommentTok{\#format = "bib")}
\end{Highlighting}
\end{Shaded}


\end{document}
